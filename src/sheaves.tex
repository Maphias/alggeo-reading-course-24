\chapter{Sheaves and presheaves}

A sheaf can be characterized as follows:

The diagram
\begin{equation}
  \begin{tikzcd}[nodes in empty cells, column sep= 1.5cm, row sep=1.5cm]
    \sheaf{F}(U) \ar[r, "\iota"] &  \prod_{i\in I} \sheaf{F}(U_i)\ar[r ,
    shift right=1, swap ] \ar[r,shift left=1] & \prod_{i,j \in I}
    \sheaf{F}(U_i \cap U_j)
  \end{tikzcd}
\end{equation}
is an equalizer where $\iota$ is a monomorphism.

We define
\begin{align*}
  \iota \colon \sheaf{F}( U)\rightarrow \prod_{i\in I}\sheaf{F}(U_i) ,
  \quad \sigma \mapsto (\sigma \at{U_i})_{i\in I},\\
  g_{k,\ell} \colon \prod_{i\in I}\sheaf{F}(U_i) \rightarrow
  \sheaf{F}(U_k\cap U_\ell),
  \quad g_{k,\ell} = \res_{U_k,U_k\cap U_\ell} \circ \pr_k.
\end{align*}

and consider the following diagram and use the universal property of
the product to obtain a map $\prod_{\ell}\prod_k g_{k,\ell}$
\begin{equation}
  \begin{tikzcd}[nodes in empty cells, column sep= 1.5cm, row sep=1.5cm]
    \prod_{i\in I}\sheaf{F}(U_i)
    \ar[dr, "\prod_k g_{k,\ell}",swap]
    \ar[r , "g_{k,\ell}"]
    \ar[d, "\prod_\ell \prod_k g_{k,\ell}",swap]
    &
    \sheaf{F}(U_k\cap U_\ell)
    \\
    \prod_{\ell \in I } \prod_{k\in I} \sheaf{F}(U_k\cap U_\ell )
    \ar[r, "\pr_\ell"]
    &
    \prod_{k\in I} \sheaf{F}(U_k \cap U_\ell )
    \ar[u, "\pr_k", swap]
  \end{tikzcd}
\end{equation}
But similarly we can define the product $\prod_\ell
\prod_k g_{\ell,k}$ by switching the indices in $g_{k,\ell}$.

This leads to two maps $\pi_{\ell,k}, \pi_{k,\ell} \colon \prod_{i\in
  I} \sheaf{F}(U_i) \rightarrow \prod_{i,j \in I}
\sheaf{F}(U_i \cap U_j)$ for which we can construct the equalizer.

Now our claim is that $\sheaf{F}(U)$ is such an equalizer. First
notice that
\begin{align*}
  \pi_{k,l}\circ \iota (\sigma) &= \pi_{k,l} (\sigma_i)_{i\in I} \\
                                &=\prod_{\ell \in I} \prod_{k\in
                                  I}(\sigma_k\at{U_\ell\cap U_k}) \\
                                &= \prod_{\ell \in I} \prod_{k\in
                                  I}(\sigma_\ell\at{U_\ell\cap U_k})\\
                                &= \pi_{\ell,k}\circ \iota (\sigma)
\end{align*}
which holds by the properties of a presheaf. It is also clear that
such a $\sigma$ is unique by the identity axiom. On the other hand any
$(\sigma_i)_{i\in I}$ such that $\pi_{k,l} (\sigma_i)_{i\in I} =
\pi_{l,k} (\sigma_i)_{i\in I}$ comes from a unique element $\sigma$ in
$\sheaf{F}(U)$ by glueing.

\section{Cofinal functors between directed sets preserve colimits}

We can generalize a directed set by a small category $I$ such that for
all $i,j\in I$ there is an object $k\in i$ such that $\Hom(i,k) \neq
\emptyset \neq \Hom(j,k)$ and such that for all $i,j$ we have $\abs{\Hom
(i,j)}\in \{0,1\}$.

We can define a cofinal functor as a functor between directed sets
considered as a category $\functor{F}\colon I\rightarrow j$ such that
for every $j\in J$ there is an $i\in I$ such that
$\Hom(j,\functor{F}(i))\neq \emptyset$.

We want to prove the following:

\begin{proposition}
  Let $I,J$ be directed sets considered as categories and
  $\functor{G}\colon J\rightarrow \category{C}$ be a diagram such that
  the colimits $\colim \functor{G}\circ \functor{F}$ and $\colim
  \functor{G}$ exist.

  If $\functor{F}\colon I\rightarrow J$ is a cofinal functor there is
  a canonical isomorphism $\colim \functor{G} \cong \colim \functor{G}
  \circ \functor{F}$.
\end{proposition}

\begin{proof}
  First notice that there by representing property of a colimit we
  have
  \begin{align*}
    \Nat(\functor{G}, \Const_c) \cong \Hom_{\category{C}}( \colim \functor{G} ,
    c ),
  \end{align*}
  where $\Const$ is the constant functor at an object in
  $\category{C}$.
  This holds especially for the identity map of $\colim
  \functor{G}$. Thus we obtain a natural transformation $\tau \colon
  \functor{G} \Rightarrow \Const_{\colim \functor{G}}$ which
  corresponds to a family of maps $\{\tau_j\colon
  \functor{G}(j)\rightarrow \colim \functor{G}\}$.

  Similarly we obtain a natural transformation $\tau'\colon
  \functor{G} \circ \functor{F} \Rightarrow \Const_{\colim \functor{G}
    \circ \functor{F}}$ corresponding to the family $\{\tau'_i \colon
  \functor{G} \circ \functor{F}(i)\rightarrow \colim \functor{G} \circ
  \functor{F} \}$

  Now we need to find isomorphisms
  \begin{align*}
    \Phi \colon \colim \functor{G} \circ \functor{F} \leftrightarrow
    \colim \functor{G} \colon \Psi.
  \end{align*}
  We do this by defining for any $j\in J$ a map $\psi_j \colon
  \functor{G}(j)\rightarrow \colim \functor{G} \circ \functor{F}$ by
  choosing an $\alpha \in \Hom (j, \functor{F}(i) ) $ for some $ i\in I$
  and setting
  \begin{align*}
    \psi_j \coloneqq \tau_{i}'\circ \functor{G}(\alpha)
  \end{align*}
  We can show that this is independent of the choice of $\alpha$ since
  for $\alpha \in \Hom(j, \functor{F}(k))$ there are maps $\sigma_{\functor{F}(k),
    \ell}\colon \functor{F}(k) \rightarrow \functor{F}(\ell)$ and $\sigma_{\functor{F}(i), \functor{F}(\ell)}$ by
  assumption on the set $J$ and the cofinality of $I$ and we get
  \begin{align*}
    \tau_{i}'\circ \functor{G} ( \alpha ) &=
                                                        \tau_{k}'\circ
                                                        \functor{G} (
                                                        \alpha' )
    \\
    \Leftrightarrow \tau_{\ell}' \circ
    \functor{G}(\functor{F}(\sigma_{i,\ell})) \circ
    \functor{G}(\alpha)
                                                      &= \tau_{\ell}' \circ
                                                        \functor{G}(\functor{F}(\sigma_{k,\ell}))\circ
                                                        \functor{G}(\alpha')
    \\
    \Leftrightarrow \tau_{\ell}' \circ
    \functor{G}(\functor{F}(\sigma_{i,\ell}) \circ\alpha)
                                                      &= \tau_{\ell}' \circ
                                                        \functor{G}(\functor{F}(\sigma_{k,\ell})\circ
                                                        \alpha')
  \end{align*}
  where we used that $\tau$ is a natural transformation to the
  constant map on $\colim G$. The equality holds since
  $\functor{F}(\sigma_{i,\ell})\circ \alpha =
  \functor{F}(\sigma_{k,\ell}) \circ\alpha'$ holds in a
  directed set.
  
  It is thus also easy to see that $\psi_j$ thus forms a natural
  transformation from $\functor{G} $ to the constant
  functor on $\colim \functor{G}\circ \functor{F}$. This induces a morphism
  \begin{align*}
    \Psi \colon \colim \functor{G} \rightarrow \colim \functor{G}\circ \functor{F}
  \end{align*}
  which is one of the maps we were looking for above.

  Now conversely we can define for all $i\in I$ we can consider the
  family of maps $\{\tau_{\functor{F}(i)}\}_{i\in I}$. These form a natural
  transformation $\functor{G}\circ \functor{F} $ to the constant
  functor on $\colim \functor{G}$. This induces a morphism
  \begin{align*}
    \Phi \colon \colim \functor{G} \circ \functor{F}\rightarrow \colim \functor{G}
  \end{align*}
  going in the other direction.
  
  To show that these are inverses, first note that $\Psi \circ \tau_j
  = \psi_j$ and
  \begin{align*}
    \Phi \circ \psi_j = \Phi \circ \tau_{\functor{F}(i)}' \circ
    \functor{G}(\alpha) =
    \tau_{\functor{F}(i)} \circ
    \functor{G}(\alpha) = \tau_j
  \end{align*}
  by the naturality of $\tau$.
  We thus consider the following commuting diagram:
  \begin{equation}
    \begin{tikzcd}[nodes in empty cells, column sep= 1.5cm, row sep=1cm]
      \functor{G}(j) \ar[r,"\tau_j"] \ar[rd, "\psi_j"]
      \ar[rdd,"\tau_j"] & \colim \functor{G} \ar[d,"\Psi"]\\
      & \colim \functor{G} \circ \functor{F} \ar[d,"\Phi"] \\
      & \colim \functor{G}.
    \end{tikzcd}
  \end{equation}
  This implies that $\Phi \circ \Psi
 \circ \tau_j = \tau_j$ for all $\tau_j$, which implies $\Phi \circ
 \Psi = \id$.
 On the other hand since $\Phi \circ \tau_i' = \tau_{\functor{F}(i)}$
 we get analogously
 \begin{align*}
   \Psi \circ \Phi \circ \tau_i' = \Psi \circ \tau_{\functor{F}(i)} =
   \psi_{\functor{F}(i)} = \tau_{i}'
 \end{align*}
 and thus $\Psi \circ \Phi = \id$ which proves the statement.
\end{proof}

%%% Local Variables:
%%% mode: latex
%%% TeX-master: "main"
%%% End:
