\chapter{Part 3}

\section{Collection of exercises on the basics of sheaves}

\begin{exercise}[2.1.A]
  See \cite[2.1.A]{vakil2024the-rising-sea}.
  Show that for the sheaf of $\Cinfty$ functions on a topological
  space denoted with $\sheaf{O}$ the stalk $\sheaf{O}_p$ is a local
  ring with maximal ideal given by
  \begin{align*}
    \mathfrak{m}_{p} = \{(f, U)\mid p\in U,\; f\in \sheaf{O}(U), \; f(p)=0\}.
  \end{align*}
\end{exercise}

\begin{proof}
  Let $I'\in \sheaf{O}_p$ be another maximal ideal not equal to
  $\mathfrak{m}_p$. Then there exists a germ $(f, U)$ such that
  $f(p)\neq 0$ in $I'$. This however is invertible, since for
  $f(p)\neq 0$ there exists $V\subset U$ such that $f(x)\neq 0$ for
  all $x\in V$. Then $(f^{-1},V)$ is a well defined germ in
  $\sheaf{O}_p$ with $(f,U)\cdot (f^{-1},V) = (1,V)$ which is a
  representative of $1\in \sheaf{O}_p$. This means that $I' =
  \sheaf{O}_p$ and we are done. 
\end{proof}

\begin{exercise}[2.2.A]
  See \cite[2.2.A]{vakil2024the-rising-sea}.
  Show that the data of a $\category{C}$ valued presheaf on a topological space $X$ is
  equivalent to a contravariant functor
  \begin{align*}
    \sheaf{F}\colon \Opens(X)^\op \rightarrow \category{C}
  \end{align*}
  from the category $\Opens(X)$ of the open sets of $X$ with morphisms
  given by the inclusion of subsets. This is especially a directed set
  and thus a nice category to work with.
\end{exercise}

\begin{proof}
  The conditions that $\res_{U,U}= \id_{\sheaf{F}(U)}$ is and
  $\res_{V,U}\circ\res_{W,V}= \res_{W,U}$ are just equivalent to the
  functoriality of above functor.
\end{proof}

\begin{exercise}[2.2.B]
  See \cite[2.2.B]{vakil2024the-rising-sea}.
  Show that the following are presheaves on $\field{C}$ (with the
  classical topology), but not sheaves:
  \begin{itemize}
  \item bounded functions,
  \item holomorphic functions admitting a holomorphic square root.
  \end{itemize}
\end{exercise}

\begin{proof}
  For the bounded functions it is clear that the bounded open subsets
  cover $\field{C}$, just consider the discs with radius $r\in
  \naturals$. On each of those the function $f(z)=z$ is bounded, but
  it does not glue to a globally bounded function.

  Similarly the function $f(z)=z$ admits a holomorphic square root for
  the two open subsets $\field{C}\setminus (0 , \infty)$ and
  $\field{C}\setminus (-\infty ,0)$. These cover $\field{C}$. But it
  does not admit a global holomorphic square root.
\end{proof}

\begin{exercise}[2.2.C]
  See \cite[2.2.C]{vakil2024the-rising-sea}.
  The identity and gluability axioms may be interpreted as saying that
  $\sheaf(\cup_{i\in I}U_i)$ is a certain limit. What is that limit?
\end{exercise}

\begin{proof}
  Gluability corresponds to the limit
  \begin{align*}
    \sheaf{F}(\cup_{i\in I}U_i) = \lim \left(
    \cup_{i,j}\sheaf{F}(U_i\cap U_j) \leftleftarrows \cup_{k\in I} \sheaf{F}(U_k) \right)
  \end{align*}
  where the two arrows are given by $\res_{U_j,U_j\cap U_i}$ and
  $\res_{U_i,U_j\cap U_i}$, see the notes for the equalizer definition
  of sheaves for more detail.
\end{proof}

\begin{exercise}[2.2.D]
  See \cite[2.2.D]{vakil2024the-rising-sea}.
  \begin{itemize}
  \item Verify that smooth functions, continuous functions, real-analytic
    funcitons, plain real-valued functions, on a manifold or Rn are
    indeed sheaves.
  \item Show that real-valued continuous functions on (open sets of) a topological space $X$ form a sheaf.
  \end{itemize}
\end{exercise}

\begin{proof}
  Trivial.
\end{proof}

\begin{exercise}[2.2.E]
  See \cite[2.2.E]{vakil2024the-rising-sea}.
  Let $\sheaf{F} (U)$ be the maps to $S$ that are locally constant,
  i.e., for any point $p$ in $U$, there is an open neighborhood of $p$
  where the function is constant. Show that this is a sheaf. (A better
  description is this: endow $S$ with the discrete topology, and let
  $\sheaf{F} (U)$ be the continuous maps $U\rightarrow S$.) This is
  called the constant sheaf (with values in $S$); do not confuse it
  with the constant presheaf. We denote this sheaf $\underline{S}$.

  \vspace{\baselineskip}
  
  First we show that the constant presheaf is not a sheaf.
\end{exercise}

\begin{proof}
  Let
  \begin{align*}
    \sheaf{F}\colon \Opens_X^{\op} \rightarrow \{S\}
  \end{align*}
  be a functor with $\abs{S}\geq 2$ defined by $\sheaf{F}(U)=S$ and
  $\res_{V,U}=\id_S$ for all $V,U$. Then this is obviously a presheaf.
  This is not a sheaf since for $x,y\in S$ and $U,V \subset X$ with $U
  \cap V = \emptyset$ the sections $x\at{U}$ and $y\at{V}$ have no
  glueing on $V\cup U$.

  \vspace{\baselineskip}

  For the second part see the next exercise.
\end{proof}


\begin{exercise}[2.2.F]
  See \cite[2.2.F]{vakil2024the-rising-sea}.
  Suppose $Y$ is a topological space. Show that ``continuous maps to $Y$''
  form a sheaf of sets on $X$. More precisely, to each open set $U$ of $X$,
  we associate the set of continuous maps of $U$ to $Y$. Show that this
  forms a sheaf.
\end{exercise}

\begin{proof}
  We define the sheaf in more detail:
  \begin{align*}
    \Fun[0](\cdot , Y) \colon \Opens_X^\op\rightarrow \Sets
  \end{align*}
  is defined by mapping an object $U\subset X$ open to $\Fun[0](U,Y)$
  and a morphism $\iota_{V,U} \colon U \rightarrow V$ in $\Opens_X^\op$ to
  $\iota^*_{V,U} \colon \Fun[0](V,Y) \rightarrow \Fun[0](U,Y)$ via
  pullback. This is a sheaf since:
  \begin{itemize}
  \item $\res_{V,U} = \iota_U^*$ fulfill the functoriality, i.e. for
    $V\subset W \subset U$ we have $\res_{U,V} = \iota_{V,U}^* = (\iota_{W,U}
    \circ \iota_{V,W})^* = \iota_{V,W}^* \circ \iota_{W,U}^* =
    \res_{W,V}\circ \res_{W,U}$ and $\res_{U,U}= \id_U^* = \id$.
  \item Identity axiom holds, since it can be checked pointwise for functions.
  \item For some open cover $\{U_i\}_{i\in I}$ of $X$ and $f_i\in
    \Fun[0](U_i,Y)$ with $f_i\at{U_i\cap U_j} = f_j \at{U_i\cap U_j}$
    for all $i,j$, we define $f\in \Fun[0](X,Y)$ by $f (p) = f_i(p)$
    for $p\in U_i$. This is well defined since for $ p \in U_i \cap
    U_j$ we have $f_i(p)=f_j(p)$.
    This is continuous, since $f^{-1}(W) = \cup_{i\in I} f_i^{-1}(W)$
    by definition for some $W\subset Y$ open.
  \end{itemize}
  We have thus checked that this indeed gives a sheaf. For $Y=S$ with
  the discrete topology this reproduces the last exercise.
\end{proof}

\begin{exercise}[2.2.G]
  See \cite[2.2.G]{vakil2024the-rising-sea}.
  This is a fancier version of the previous exercise.
  \begin{itemize}
  \item (sheaf of sections of a map) Suppose we are given a
    continuous map $\mu \colon Y \rightarrow X$. Show that ``sections of $\mu$'' form a
    sheaf. More precisely, to each open set $U$ of $X$, associate the set
    of continuous maps $s \colon U \rightarrow Y$ such that $\mu\circ
    s = \id\at{U}$. Show that
    this forms a sheaf. (For those who have heard of vector bundles,
    these are a good example.) This is motivation for the phrase
    ``section of a sheaf''.
  \item (This exercise is for those who know what a topological group
    is. If you don’t know what a topological group is, you might be
    able to guess.) Suppose that $Y$ is a topological group. Show that
    continuous maps to $Y$ form a sheaf of groups.
  \end{itemize}
\end{exercise}

\begin{proof}
  We define the following sheaf:
  \begin{align*}
    \Gamma_s(\cdot) \colon \Opens_X^\op \rightarrow \Sets
  \end{align*}
  by setting
  \begin{align*}
    \Gamma_\mu(U)= \{f\in \Fun[0](U,Y)\mid \mu\circ f = \id\at{U}\}
  \end{align*}
  and mapping an inclusion $\iota_{V,U} \colon U \rightarrow V$ in $\Opens_X^\op$ to
  $\iota^*_{V,U} \colon \Fun[0](V,Y) \rightarrow \Fun[0](U,Y)$ via
  pullback.
  We need to show that this is well defined. We see this by
  calculating $\mu \circ \res_{U,V}f = \mu \circ \iota_{V,U}^* f = \mu \circ f \at{V} = \id
  \at{V}$ for $f\in \Gamma_\mu (U)$.
  The functoriality follows analogously to the previous exercise.

  The identity axiom also holds by equal argument.

  For the gluability it remains to see that $f$ defined as before is
  contained in $\Gamma_\mu(X)$. This can be checked pointwise:

  For all $p\in X$ there is a $U_i$ in the open cover such that $p\in
  U_i$. But then $\mu \circ f (p)= \mu \circ f \at{U_i} (p) = \mu
  \circ f_i (p)= \id\at{U_i}(p) = p$. Since this holds for all $p\in
  X$ we are done.

  \vspace{\baselineskip}

  For the second part we can consider the same sheaf as in the
  previous exercise.

  We define the group operation by
  \begin{align*}
    m\colon \Fun[0](U,Y)\times \Fun[0](U,Y) \rightarrow \Fun[0](U,Y)
    ,\quad (f(p),g(p))\mapsto f(p)\cdot g(p)
  \end{align*}
  which is the pointwise operation. This is obviously well defined. It
  remains to show that inverse exist. This is the case since for $f\in
  \Fun[0](U,Y)$ we can define $f^{-1}\in \Fun[0](U,Y)$ by $f^{-1}(p)=
  (f(p))^{-1}$. This is continuous since forming inverses is
  continuous. The neutral element is obviously given by the constant
  map to the group unit.
\end{proof}

\begin{exercise}[2.2.H]
  See \cite[2.2.H]{vakil2024the-rising-sea}.
  Suppose $\pi \colon X \rightarrow Y$ is a continuous map, and
  $\sheaf{F}$ is a presheaf on $X$. Then define $\pi_*\sheaf{F}$ by
  $\pi_*\sheaf{F} (V) =\sheaf{F}(\pi^{-1}(V))$, where $V$ is an open
  subset of $Y$. Show that $\pi_*\sheaf{F}$ is a presheaf on $Y$, and
  is a sheaf if $\sheaf{F}$ is. This is called the pushforward or
  direct image of $\sheaf{F}$. More precisely, $\pi_*F$ is called the pushforward of $\sheaf{F}$ by $\pi$.
\end{exercise}

\begin{proof}
  We define $\pi_*\res_{U,V}\colon \pi_*\sheaf{F}(U)\rightarrow
  \pi_*\sheaf{F}(V)$ by $\pi_*\res_{U,V}=
  \res_{\pi^{-1}(U),\pi^{-1}(V)}$. This is well defined, since for
  $V\subset U$ also $\pi^{-1}(V)\subset \pi^{-1}(U)$. Then we fulfill
  the conditions for a presheaf since $\res$ fulfills these
  conditions.

  To show that it is even a sheaf, we check identity and glueability:
  For identity it is clear that for a cover $\{W_i\}_{i\in I}$ of $Y$
  the set $\{U_i\coloneqq \pi^{-1}(W_i)\}_{i\in I}$ is an open cover
  of $X$. Let $f,g\in \pi_*\sheaf{F}(Y)$ with $\pi_*\res_{Y,U_i}f=
  \pi_*\res_{Y,W_i}g$ for all $W_i$. Then $f\in \sheaf{F}(X)$ with
  $\res_{X,U_i}f= \res_{X,U_i}g$. Since the $U_i$ are an open cover $f
  = g \in \sheaf{F}(X)$ follows, which implies that $f=g \in
  \pi_*^{-1}\sheaf{F}(Y)$.

  For the gluability pick $f_i \in \pi_*\sheaf{F}(W_i)$ with
  $\pi_*\res_{W_i,W_i\cap W_j} f_i = \pi_*\res_{W_j,W_i\cap
    W_j}f_j$. This means that for $f_i\in \sheaf{F}(U_i)$ we have
  $\res_{U_i,U_i\cap U_j}f_i = \res_{U_j,U_i\cap U_j}f_j$. Since
  $\sheaf{F}$ is a sheaf, we can glue the $f_i$ to get $f\in
  \sheaf{F}(X)$ which corresponds to $f\in \pi_*\sheaf{F}(Y)$ with
  $\pi_*\res_{Y,W_i}f = \res_{X,U_i}f = f_i$ by construction.
  Thus $\pi_*\sheaf{F}$ is a sheaf if $\sheaf{F}$ is.
\end{proof}

%%% Local Variables:
%%% mode: latex
%%% TeX-master: "main"
%%% End:
