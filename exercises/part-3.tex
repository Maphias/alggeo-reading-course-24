\chapter{Part 3}

\section{Collection of exercises on the basics of sheaves}

\begin{exercise}[2.1.A]
  See \cite[2.1.A]{vakil2024the-rising-sea}.
  Show that for the sheaf of $\Cinfty$ functions on a topological
  space denoted with $\sheaf{O}$ the stalk $\sheaf{O}_p$ is a local
  ring with maximal ideal given by
  \begin{align*}
    \mathfrak{m}_{p} = \{(f, U)\mid p\in U,\; f\in \sheaf{O}(U), \; f(p)=0\}.
  \end{align*}
\end{exercise}

\begin{proof}
  Let $I'\in \sheaf{O}_p$ be another maximal ideal not equal to
  $\mathfrak{m}_p$. Then there exists a germ $(f, U)$ such that
  $f(p)\neq 0$ in $I'$. This however is invertible, since for
  $f(p)\neq 0$ there exists $V\subset U$ such that $f(x)\neq 0$ for
  all $x\in V$. Then $(f^{-1},V)$ is a well defined germ in
  $\sheaf{O}_p$ with $(f,U)\cdot (f^{-1},V) = (1,V)$ which is a
  representative of $1\in \sheaf{O}_p$. This means that $I' =
  \sheaf{O}_p$ and we are done. 
\end{proof}

\begin{exercise}[2.2.A]
  See \cite[2.2.A]{vakil2024the-rising-sea}.
  Show that the data of a $\category{C}$ valued presheaf on a topological space $X$ is
  equivalent to a contravariant functor
  \begin{align*}
    \sheaf{F}\colon \Opens(X)^\op \rightarrow \category{C}
  \end{align*}
  from the category $\Opens(X)$ of the open sets of $X$ with morphisms
  given by the inclusion of subsets. This is especially a directed set
  and thus a nice category to work with.
\end{exercise}

\begin{proof}
  The conditions that $\res_{U,U}= \id_{\sheaf{F}(U)}$ is and
  $\res_{V,U}\circ\res_{W,V}= \res_{W,U}$ are just equivalent to the
  functoriality of above functor.
\end{proof}

\begin{exercise}[2.2.B]
  See \cite[2.2.B]{vakil2024the-rising-sea}.
  Show that the following are presheaves on $\field{C}$ (with the
  classical topology), but not sheaves:
  \begin{itemize}
  \item bounded functions,
  \item holomorphic functions admitting a holomorphic square root.
  \end{itemize}
\end{exercise}

\begin{proof}
  For the bounded functions it is clear that the bounded open subsets
  cover $\field{C}$, just consider the discs with radius $r\in
  \naturals$. On each of those the function $f(z)=z$ is bounded, but
  it does not glue to a globally bounded function.

  Similarly the function $f(z)=z$ admits a holomorphic square root for
  the two open subsets $\field{C}\setminus (0 , \infty)$ and
  $\field{C}\setminus (-\infty ,0)$. These cover $\field{C}$. But it
  does not admit a global holomorphic square root.
\end{proof}

\begin{exercise}[2.2.C]
  See \cite[2.2.C]{vakil2024the-rising-sea}.
  The identity and gluability axioms may be interpreted as saying that
  $\sheaf(\cup_{i\in I}U_i)$ is a certain limit. What is that limit?
\end{exercise}

\begin{proof}
  Gluability corresponds to the limit
  \begin{align*}
    \sheaf{F}(\cup_{i\in I}U_i) = \lim \left(
    \cup_{i,j}\sheaf{F}(U_i\cap U_j) \leftleftarrows \cup_{k\in I} \sheaf{F}(U_k) \right)
  \end{align*}
  where the two arrows are given by $\res_{U_j,U_j\cap U_i}$ and
  $\res_{U_i,U_j\cap U_i}$, see the notes for the equalizer definition
  of sheaves for more detail.
\end{proof}

\begin{exercise}[2.2.D]
  See \cite[2.2.D]{vakil2024the-rising-sea}.
  \begin{itemize}
  \item Verify that smooth functions, continuous functions, real-analytic
    funcitons, plain real-valued functions, on a manifold or Rn are
    indeed sheaves.
  \item Show that real-valued continuous functions on (open sets of) a topological space $X$ form a sheaf.
  \end{itemize}
\end{exercise}

\begin{proof}
  Trivial.
\end{proof}



%%% Local Variables:
%%% mode: latex
%%% TeX-master: "main"
%%% End:
