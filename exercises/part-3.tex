\chapter{Part 3}

\section{Collection of exercises on the basics of sheaves}

\begin{exercise}[2.1.A]
  See \cite[2.1.A]{vakil2024the-rising-sea}.
  Show that for the sheaf of $\Cinfty$ functions on a topological
  space denoted with $\sheaf{O}$ the stalk $\sheaf{O}_p$ is a local
  ring with maximal ideal given by
  \begin{align*}
    \mathfrak{m}_{p} = \{(f, U)\mid p\in U,\; f\in \sheaf{O}(U), \; f(p)=0\}.
  \end{align*}
\end{exercise}

\begin{proof}
  Let $I'\in \sheaf{O}_p$ be another maximal ideal not equal to
  $\mathfrak{m}_p$. Then there exists a germ $(f, U)$ such that
  $f(p)\neq 0$ in $I'$. This however is invertible, since for
  $f(p)\neq 0$ there exists $V\subset U$ such that $f(x)\neq 0$ for
  all $x\in V$. Then $(f^{-1},V)$ is a well defined germ in
  $\sheaf{O}_p$ with $(f,U)\cdot (f^{-1},V) = (1,V)$ which is a
  representative of $1\in \sheaf{O}_p$. This means that $I' =
  \sheaf{O}_p$ and we are done. 
\end{proof}

\begin{exercise}[2.2.A]
  See \cite[2.2.A]{vakil2024the-rising-sea}.
  Show that the data of a $\category{C}$ valued presheaf on a topological space $X$ is
  equivalent to a contravariant functor
  \begin{align*}
    \sheaf{F}\colon \Opens(X)^\op \rightarrow \category{C}
  \end{align*}
  from the category $\Opens(X)$ of the open sets of $X$ with morphisms
  given by the inclusion of subsets. This is especially a directed set
  and thus a nice category to work with.
\end{exercise}

\begin{proof}
  The conditions that $\res_{U,U}= \id_{\sheaf{F}(U)}$ is and
  $\res_{V,U}\circ\res_{W,V}= \res_{W,U}$ are just equivalent to the
  functoriality of above functor.
\end{proof}

\begin{exercise}[2.2.B]
  See \cite[2.2.B]{vakil2024the-rising-sea}.
  Show that the following are presheaves on $\field{C}$ (with the
  classical topology), but not sheaves:
  \begin{itemize}
  \item bounded functions,
  \item holomorphic functions admitting a holomorphic square root.
  \end{itemize}
\end{exercise}

\begin{proof}
  For the bounded functions it is clear that the bounded open subsets
  cover $\field{C}$, just consider the discs with radius $r\in
  \naturals$. On each of those the function $f(z)=z$ is bounded, but
  it does not glue to a globally bounded function.

  Similarly the function $f(z)=z$ admits a holomorphic square root for
  the two open subsets $\field{C}\setminus (0 , \infty)$ and
  $\field{C}\setminus (-\infty ,0)$. These cover $\field{C}$. But it
  does not admit a global holomorphic square root.
\end{proof}

\begin{exercise}[2.2.C]
  See \cite[2.2.C]{vakil2024the-rising-sea}.
  The identity and gluability axioms may be interpreted as saying that
  $\sheaf(\cup_{i\in I}U_i)$ is a certain limit. What is that limit?
\end{exercise}

\begin{proof}
  Gluability corresponds to the limit
  \begin{align*}
    \sheaf{F}(\cup_{i\in I}U_i) = \lim \left(
    \cup_{i,j}\sheaf{F}(U_i\cap U_j) \leftleftarrows \cup_{k\in I} \sheaf{F}(U_k) \right)
  \end{align*}
  where the two arrows are given by $\res_{U_j,U_j\cap U_i}$ and
  $\res_{U_i,U_j\cap U_i}$, see the notes for the equalizer definition
  of sheaves for more detail.
\end{proof}

\begin{exercise}[2.2.D]
  See \cite[2.2.D]{vakil2024the-rising-sea}.
  \begin{itemize}
  \item Verify that smooth functions, continuous functions, real-analytic
    funcitons, plain real-valued functions, on a manifold or Rn are
    indeed sheaves.
  \item Show that real-valued continuous functions on (open sets of) a topological space $X$ form a sheaf.
  \end{itemize}
\end{exercise}

\begin{proof}
  Trivial.
\end{proof}

\begin{exercise}[2.2.E]
  See \cite[2.2.E]{vakil2024the-rising-sea}.
  Let $\sheaf{F} (U)$ be the maps to $S$ that are locally constant,
  i.e., for any point $p$ in $U$, there is an open neighborhood of $p$
  where the function is constant. Show that this is a sheaf. (A better
  description is this: endow $S$ with the discrete topology, and let
  $\sheaf{F} (U)$ be the continuous maps $U\rightarrow S$.) This is
  called the constant sheaf (with values in $S$); do not confuse it
  with the constant presheaf. We denote this sheaf $\underline{S}$.

  \vspace{\baselineskip}
  
  First we show that the constant presheaf is not a sheaf.
\end{exercise}

\begin{proof}
  Let
  \begin{align*}
    \sheaf{F}\colon \Opens_X^{\op} \rightarrow \{S\}
  \end{align*}
  be a functor with $\abs{S}\geq 2$ defined by $\sheaf{F}(U)=S$ and
  $\res_{V,U}=\id_S$ for all $V,U$. Then this is obviously a presheaf.
  This is not a sheaf since for $x,y\in S$ and $U,V \subset X$ with $U
  \cap V = \emptyset$ the sections $x\at{U}$ and $y\at{V}$ have no
  glueing on $V\cup U$.

  \vspace{\baselineskip}

  For the second part see the next exercise.
\end{proof}


\begin{exercise}[2.2.F]
  See \cite[2.2.F]{vakil2024the-rising-sea}.
  Suppose $Y$ is a topological space. Show that ``continuous maps to $Y$''
  form a sheaf of sets on $X$. More precisely, to each open set $U$ of $X$,
  we associate the set of continuous maps of $U$ to $Y$. Show that this
  forms a sheaf.
\end{exercise}

\begin{proof}
  We define the sheaf in more detail:
  \begin{align*}
    \Fun[0](\cdot , Y) \colon \Opens_X^\op\rightarrow \Sets
  \end{align*}
  is defined by mapping an object $U\subset X$ open to $\Fun[0](U,Y)$
  and a morphism $\iota_{V,U} \colon U \rightarrow V$ in $\Opens_X^\op$ to
  $\iota^*_{V,U} \colon \Fun[0](V,Y) \rightarrow \Fun[0](U,Y)$ via
  pullback. This is a sheaf since:
  \begin{itemize}
  \item $\res_{V,U} = \iota_U^*$ fulfill the functoriality, i.e. for
    $V\subset W \subset U$ we have $\res_{U,V} = \iota_{V,U}^* = (\iota_{W,U}
    \circ \iota_{V,W})^* = \iota_{V,W}^* \circ \iota_{W,U}^* =
    \res_{W,V}\circ \res_{W,U}$ and $\res_{U,U}= \id_U^* = \id$.
  \item Identity axiom holds, since it can be checked pointwise for functions.
  \item For some open cover $\{U_i\}_{i\in I}$ of $X$ and $f_i\in
    \Fun[0](U_i,Y)$ with $f_i\at{U_i\cap U_j} = f_j \at{U_i\cap U_j}$
    for all $i,j$, we define $f\in \Fun[0](X,Y)$ by $f (p) = f_i(p)$
    for $p\in U_i$. This is well defined since for $ p \in U_i \cap
    U_j$ we have $f_i(p)=f_j(p)$.
    This is continuous, since $f^{-1}(W) = \cup_{i\in I} f_i^{-1}(W)$
    by definition for some $W\subset Y$ open.
  \end{itemize}
  We have thus checked that this indeed gives a sheaf. For $Y=S$ with
  the discrete topology this reproduces the last exercise.
\end{proof}

\begin{exercise}[2.2.G]
  See \cite[2.2.G]{vakil2024the-rising-sea}.
  This is a fancier version of the previous exercise.
  \begin{itemize}
  \item (sheaf of sections of a map) Suppose we are given a
    continuous map $\mu \colon Y \rightarrow X$. Show that ``sections of $\mu$'' form a
    sheaf. More precisely, to each open set $U$ of $X$, associate the set
    of continuous maps $s \colon U \rightarrow Y$ such that $\mu\circ
    s = \id\at{U}$. Show that
    this forms a sheaf. (For those who have heard of vector bundles,
    these are a good example.) This is motivation for the phrase
    ``section of a sheaf''.
  \item (This exercise is for those who know what a topological group
    is. If you don’t know what a topological group is, you might be
    able to guess.) Suppose that $Y$ is a topological group. Show that
    continuous maps to $Y$ form a sheaf of groups.
  \end{itemize}
\end{exercise}

\begin{proof}
  We define the following sheaf:
  \begin{align*}
    \Gamma_s(\cdot) \colon \Opens_X^\op \rightarrow \Sets
  \end{align*}
  by setting
  \begin{align*}
    \Gamma_\mu(U)= \{f\in \Fun[0](U,Y)\mid \mu\circ f = \id\at{U}\}
  \end{align*}
  and mapping an inclusion $\iota_{V,U} \colon U \rightarrow V$ in $\Opens_X^\op$ to
  $\iota^*_{V,U} \colon \Fun[0](V,Y) \rightarrow \Fun[0](U,Y)$ via
  pullback.
  We need to show that this is well defined. We see this by
  calculating $\mu \circ \res_{U,V}f = \mu \circ \iota_{V,U}^* f = \mu \circ f \at{V} = \id
  \at{V}$ for $f\in \Gamma_\mu (U)$.
  The functoriality follows analogously to the previous exercise.

  The identity axiom also holds by equal argument.

  For the gluability it remains to see that $f$ defined as before is
  contained in $\Gamma_\mu(X)$. This can be checked pointwise:

  For all $p\in X$ there is a $U_i$ in the open cover such that $p\in
  U_i$. But then $\mu \circ f (p)= \mu \circ f \at{U_i} (p) = \mu
  \circ f_i (p)= \id\at{U_i}(p) = p$. Since this holds for all $p\in
  X$ we are done.

  \vspace{\baselineskip}

  For the second part we can consider the same sheaf as in the
  previous exercise.

  We define the group operation by
  \begin{align*}
    m\colon \Fun[0](U,Y)\times \Fun[0](U,Y) \rightarrow \Fun[0](U,Y)
    ,\quad (f(p),g(p))\mapsto f(p)\cdot g(p)
  \end{align*}
  which is the pointwise operation. This is obviously well defined. It
  remains to show that inverse exist. This is the case since for $f\in
  \Fun[0](U,Y)$ we can define $f^{-1}\in \Fun[0](U,Y)$ by $f^{-1}(p)=
  (f(p))^{-1}$. This is continuous since forming inverses is
  continuous. The neutral element is obviously given by the constant
  map to the group unit.
\end{proof}

\begin{exercise}[2.2.H]
  See \cite[2.2.H]{vakil2024the-rising-sea}.
  Suppose $\pi \colon X \rightarrow Y$ is a continuous map, and
  $\sheaf{F}$ is a presheaf on $X$. Then define $\pi_*\sheaf{F}$ by
  $\pi_*\sheaf{F} (V) =\sheaf{F}(\pi^{-1}(V))$, where $V$ is an open
  subset of $Y$. Show that $\pi_*\sheaf{F}$ is a presheaf on $Y$, and
  is a sheaf if $\sheaf{F}$ is. This is called the pushforward or
  direct image of $\sheaf{F}$. More precisely, $\pi_*F$ is called the pushforward of $\sheaf{F}$ by $\pi$.
\end{exercise}

\begin{proof}
  We define $\pi_*\res_{U,V}\colon \pi_*\sheaf{F}(U)\rightarrow
  \pi_*\sheaf{F}(V)$ by $\pi_*\res_{U,V}=
  \res_{\pi^{-1}(U),\pi^{-1}(V)}$. This is well defined, since for
  $V\subset U$ also $\pi^{-1}(V)\subset \pi^{-1}(U)$. Then we fulfill
  the conditions for a presheaf since $\res$ fulfills these
  conditions.

  To show that it is even a sheaf, we check identity and glueability:
  For identity it is clear that for a cover $\{W_i\}_{i\in I}$ of $Y$
  the set $\{U_i\coloneqq \pi^{-1}(W_i)\}_{i\in I}$ is an open cover
  of $X$. Let $f,g\in \pi_*\sheaf{F}(Y)$ with $\pi_*\res_{Y,U_i}f=
  \pi_*\res_{Y,W_i}g$ for all $W_i$. Then $f\in \sheaf{F}(X)$ with
  $\res_{X,U_i}f= \res_{X,U_i}g$. Since the $U_i$ are an open cover $f
  = g \in \sheaf{F}(X)$ follows, which implies that $f=g \in
  \pi_*^{-1}\sheaf{F}(Y)$.

  For the gluability pick $f_i \in \pi_*\sheaf{F}(W_i)$ with
  $\pi_*\res_{W_i,W_i\cap W_j} f_i = \pi_*\res_{W_j,W_i\cap
    W_j}f_j$. This means that for $f_i\in \sheaf{F}(U_i)$ we have
  $\res_{U_i,U_i\cap U_j}f_i = \res_{U_j,U_i\cap U_j}f_j$. Since
  $\sheaf{F}$ is a sheaf, we can glue the $f_i$ to get $f\in
  \sheaf{F}(X)$ which corresponds to $f\in \pi_*\sheaf{F}(Y)$ with
  $\pi_*\res_{Y,W_i}f = \res_{X,U_i}f = f_i$ by construction.
  Thus $\pi_*\sheaf{F}$ is a sheaf if $\sheaf{F}$ is.
\end{proof}

\begin{exercise}[2.2.I]
  See \cite[2.2.I]{vakil2024the-rising-sea}.
  Suppose $\pi \colon X \rightarrow Y$ is a continuous map, and $\sheaf{F}$ is a sheaf of sets (or rings or $\ring{A}$-modules) on $X$. If $\pi(p) = q$, describe the natural morphism of stalks $(\pi_*\sheaf{F} )_q \rightarrow F_p$.
\end{exercise}

\begin{proof}
  Using explicit forms of stalks we get
  \begin{align*}
    (\pi_*\sheaf{F})_q=\{(g,V)\mid q\in V, \; g\in \pi_*\sheaf{F}(V)\}/\sim
  \end{align*}
  and we map this to
  \begin{align*}
    \{(g\circ \pi,\pi^{-1}(V))\mid q\in V,\mid g\in \pi_*g\in
    \pi_*\sheaf{F}(V)\}/\sim \subset \sheaf{F}_p.
  \end{align*}
  This is well defined since $f\sim g$ in $(\pi_*\sheaf{F})_q$ means
  by continuity that there is $\pi^{-1}(W)\subset X$ with $q\in W$
  such that $f\circ \pi = g\circ \pi$ on $\pi^{-1}(W)$.

  For the universal approach we consider $\sheaf{F}_x = \colim_x
  \sheaf{F}(U)$ where the colimit is taken over the directed set of
  open subsets containing $x$.
  Then we can consider
  \begin{equation}
    \begin{tikzcd}[nodes in empty cells, column sep= 1.5cm, row sep=1.5cm]
      \colim_q \pi_*\sheaf{F}(V)
      \ar[r,"="]
      \ar[rr,bend left=20,"\exists!"]
      &
      \colim_q \sheaf{F}(\pi^{-1}(V)) \ar[r,
      hook ]
      &
      \colim_p \sheaf{F}(U)
    \end{tikzcd}
  \end{equation}
  since the condition that $\pi$ is continuous is equivalent to the
  condition that the subsets $\pi^{-1}(V)$ for neighbourhoods of $q$
  are a cofinal subset of the neighbourhood basis of $p$, i.e. for all
  $q\in W$ there exists $p\in V$ such that $\pi (V) \subset W$.

  Then we could also derive this result since cofinal functors
  preserve colimits, see notes.
\end{proof}

\begin{exercise}[2.2.J]
  See \cite[2.2.J]{vakil2024the-rising-sea}.
  If $(X, \sheaf{O}_X)$ is a ringed space, and $\sheaf{F}$ is an
  $\sheaf{F}_X$-module, describe how for each $p \in X$, $\sheaf{F}_p$ is an $\sheaf{O}_{X,p}$-module.
\end{exercise}

\begin{proof}
  We can take the limit over the following diagram
  \begin{equation}
    \begin{tikzcd}[nodes in empty cells, column sep= 1.5cm, row sep=1.0cm]
      \sheaf{O}_X(V)\times \sheaf{F}(V)
      \ar[r, "\acts"]
      \ar[d, "\res_{V,U}\times \res_{V,U}"]
      &
      \sheaf{F}(V)
      \ar[d,"\res_{V,U}"]
      \\
      \sheaf{O}_X(U)\times \sheaf{F}(U)
      \ar[r, "\acts"]
      \ar[d, "\res_{U,W}\times \res_{U,W}"]
      &
      \sheaf{F}(U)
      \ar[d,"\res_{U,W}"]
      \\
      \sheaf{O}_X(W)\times \sheaf{F}(W)
      \ar[r, "\acts"]
      \ar[d,dashed]
      &
      \sheaf{F}(W)
      \ar[d, dashed]
      \\
      \colim_p \sheaf{O}_X\times \sheaf{F}
      \ar[r,"\exists !"]
      &
      \colim_p \sheaf{F}
    \end{tikzcd}
  \end{equation}
  Then the action of $\sheaf{O}_{X,p}$ is induced by the lowest
  map. Explicitly we define $\sigma \acts f$ for $\sigma \in
  \sheaf{O}_p$ and $f\in \sheaf{F}_p$ by defining for two
  representatives $(\sigma,V)$ and $(f,U)$ on the restriction
  $(\sigma,V)\acts (f,U) \coloneqq (\sigma\at{U\cap V}\acts f\at{U\cap
    V}, U\cap V)$. This is well defined, since the action commutes with
  the restriction maps.
\end{proof}

\begin{exercise}[2.3.A]
  See \cite[2.3.A]{vakil2024the-rising-sea}.
  If $\phi \colon \sheaf{F}\rightarrow \sheaf{G}$ is a morphism of presheaves on $X$, and $p \in X$, describe an induced morphism of stalks $\phi_p \colon \sheaf{F}_p \rightarrow \sheaf{G}_p$.
\end{exercise}

\begin{proof}
  Again by the property that $\phi$ is a natural transformation, the
  following diagram commutes:
  \begin{equation}
    \begin{tikzcd}[nodes in empty cells, column sep= 1.5cm, row sep=1.0cm]
      \sheaf{F}(V)
      \ar[r, "\phi"]
      \ar[d, "\res_{V,U}"]
      &
      \sheaf{G}(V)
      \ar[d,"\res_{V,U}"]
      \\
      \sheaf{F}(U)
      \ar[r, "\phi"]
      \ar[d, "\res_{U,W}"]
      &
      \sheaf{G}(U)
      \ar[d,"\res_{U,W}"]
      \\
      \sheaf{F}(W)
      \ar[r, "\phi"]
      \ar[d,dashed]
      &
      \sheaf{G}(W)
      \ar[d, dashed]
      \\
      \colim_p \sheaf{F}
      \ar[r,"\exists !\; \phi_p"]
      &
      \colim_p \sheaf{G}
    \end{tikzcd}
  \end{equation}
  and thus for $\tau_U \colon \sheaf{G}(U)\rightarrow \colim_p
  \sheaf{G}= \sheaf{G}_p$ the maps $\tau_U\circ \phi$ form a cone
  $\{\tau_U\circ \phi \colon \sheaf{F}(U)\rightarrow
  \sheaf{G}_p\}_{p\in U}$ which induces the morphism $\phi_p \colon
  \sheaf{F}_p \rightarrow \sheaf{G}_p$ by universal property.
\end{proof}

\begin{exercise}[2.3.B]
  See \cite[2.3.B]{vakil2024the-rising-sea}.
  Suppose $\pi\colon  X \rightarrow Y$ is a continuous map of topological spaces (i.e., a morphism in the category of topological spaces). Show that pushforward gives a functor $\pi_*\colon \Sets_X \rightarrow \Sets_Y$. Here Sets can be replaced by other categories.
\end{exercise}

\begin{proof}
  It is clear that $\pi_*\sheaf{F}(U)$ is a sheaf. It remains to show,
  that for morphisms of $\Sets_X$ given by $\phi\colon \sheaf{F}\rightarrow
  \sheaf{G}$ and $\psi \colon \sheaf{G}\rightarrow \sheaf{H}$ we have
  for the composition
  \begin{align}
    \pi_*(\psi \circ \phi) = \pi_*(\psi)\circ \pi_*(\phi).
  \end{align}
  We define this functor on a morphism $\{\phi_U\colon
  \sheaf{F}(U)\rightarrow \sheaf{G}(U)\}_{U \in \Opens_X}$ by
  \begin{align}
    \pi_*\phi \coloneqq \{\pi_*\phi_V \coloneqq
    \phi_{\pi^{-1}(V)}\colon \pi_*\sheaf{F}(V)\rightarrow
    \pi_*\sheaf{G}(V)\}_{V\in \Opens_Y}.
  \end{align}
  Then the functoriality is given by the following diagram:
  \begin{equation}
    \begin{tikzcd}[nodes in empty cells, column sep= 1.5cm, row sep=1.5cm]
      \pi_*\sheaf{F}(V) = \sheaf{F}(\pi^{-1}(V))
      \ar[r,"\phi_{\pi^{-1}(V)}"]
      \ar[d,"\pi_*\res_{V,W} = \res_{\pi^{-1}(V),\pi^{-1}(W)}"]
      &
      \pi_*\sheaf{G}(V) = \sheaf{G}(\pi^{-1}(V))
      \ar[r,"\psi_{\pi^{-1}(V)}"]
      \ar[d,"\pi_*\res_{V,W}= \res_{\pi^{-1}(V),\pi^{-1}(W)}"]
      &
      \pi_*\sheaf{H}(V) = \sheaf{H}(\pi^{-1}(V))
      \ar[d,"\pi_*\res_{V,W}= \res_{\pi^{-1}(V),\pi^{-1}(W)}"]
      \\
      \pi_*\sheaf{F}(W) = \sheaf{F}(\pi^{-1}(W))
      \ar[r,"\phi_{\pi^{-1}(V)}"]
      &
      \pi_*\sheaf{G}(W) = \sheaf{G}(\pi^{-1}(W))
      \ar[r,"\psi_{\pi^{-1}(V)}"]
      &
      \pi_*\sheaf{H}(W) = \sheaf{H}(\pi^{-1}(W))
    \end{tikzcd}
  \end{equation}
  and the fact that $\id_* \id_{\sheaf{F}} = \id_{\sheaf{F}}$ by construction. 
\end{proof}

\begin{exercise}[2.3.C]
  See \cite[2.3.C]{vakil2024the-rising-sea}.
  Suppose $\sheaf{F}$ and $\sheaf{G}$ are two sheaves of sets on $X$. (In fact, it will
  suffice that $\sheaf{F}$ is a presheaf.) Let $\Hom(\sheaf{F} , \sheaf{G} )$ be the collection of
  data
  \begin{align*}
    \Hom(\sheaf{F} , \sheaf{G} )(U) \coloneqq \Hom(\sheaf{F}\at{U},
    \sheaf{G}\at{U}).
  \end{align*}
  (Recall the notation $\sheaf{G} \at{U}$, the restriction of the
  sheaf to the open set $U$) Show that this is a sheaf of sets on
  $X$. This sheaf is called ``sheaf Hom''.
\end{exercise}

\begin{proof}
  Let $\sheaf{F}$ be a presheaf and $\sheaf{G}$ a sheaf. We define the
  restriction maps for $V\subset U$ by
  \begin{align*}
    \res_{U,V} \colon \Hom(\sheaf{F} , \sheaf{G} )(U) \rightarrow
    \Hom(\sheaf{F} , \sheaf{G} )(V), \quad \{\phi_W\}_{W\in \Opens_U} \mapsto
    \{\phi_W\}_{W\in \Opens_V}
  \end{align*}
  This is well defined, since $ \{\phi_W\}_{W\in \Opens_V}$ is still a
  natural transformation $\sheaf{F}\at{V}\rightarrow \sheaf{G}\at{V}$.

  Now for the functoriality it is clear that this holds since we just
  make our set of morphisms in the natural transformation smaller,
  i.e. forgetting stuff is functorial.

  For the identity we consider a covering $\{U_i\}_{i\in I}$ of
  $X$. Then for two natural transformations $f,g\colon \sheaf{F}
  \rightarrow \sheaf{G}$ such that $\{f_W\}_{W\in \Opens_{U_i}} =
  \{f_W\}_{W\in \Opens_{U_i}}$ we need to show that for any open set
  $V\subset X$ we have $f_V = g_V$. Then the set $\{U_i\cap V\}_{\i\in
    I}$ is an open covering of $U$ and $\res_{V,U_i\cap V}f_V =
  \res_{V,U_i\cap V}g_V$ since we assumed that $\{f_W\}_{W\in \Opens_{U_i}} =
  \{f_W\}_{W\in \Opens_{U_i}}$ and $U_i\cap V$ is open in $U_i$. Thus, since $\sheaf{G}\at{V}$ is a sheaf,
  we have $f_V = g_V$, which means that $\{f_V\}_{V\in \Opens_X}=
  \{g_V\}_{V\in \Opens_X}$.

  For the gluability we consider the set $\{\{f_W^i\}_{W\in
    \Opens_{U_i}}\}_{i\in I}$ such that
  \begin{align*}
    \{f^i_W\}_{W\in \Opens_{U_i\cap
    U_j}} = \{f^j_W\}_{W\in \Opens_{U_i\cap
    U_j}}.
  \end{align*}
  We have to construct a global
  $\{f_V\}_{V\in \Opens_X}$, i.e. we need to define an $f_V$ for all
  open subsets of $X$. It is clear again that $\{U_i\cap V\}_{i\in I}$
  is an open cover of $V$. Then the maps $\{f^i_{U_i\cap V}\}_{i\in
    I}$ fulfill the condition that on $U_i \cap U_j \cap V$ we have
  $f^i_{U_i\cap V}\at{U_j \cap V} = f^i_{U_i\cap U_j\cap V} =
  f^j_{U_i\cap U_j\cap V} = f^j_{U_j \cap V}\at{U_i}$ and we can thus
  glue these to a map $f\at{V}$. Then the set $\{f_V\}_{V\in
    \Opens_X}$ fulfills the demands.  
\end{proof}

\begin{exercise}[2.3.D]
  See \cite[2.3.D]{vakil2024the-rising-sea}.
  \begin{itemize}
  \item If $\sheaf{F}$ is a sheaf of sets on $X$, then show that
    $\Hom(\{p\}, \sheaf{F} ) \cong \sheaf{F}$, where $\{p\}$ is the
    constant sheaf “with values in the one element set $\{p\}$”.
  \item If $\sheaf{F}$ is a sheaf of abelian groups on $X$, then
    show that $\Hom_{\Ab} (\underline{Z}, \sheaf{F} )\cong
    \sheaf{F}$ (an isomorphism of sheaves of abelian groups).
  \item If $\sheaf{F}$ is an $\sheaf{O}_X$-module, then show that
    $\Hom_{\Modules_{\sheaf{O}_X}} (\sheaf{O}_X,\sheaf{F} ) \cong
    \sheaf{F}$ (an isomorphism of OX-modules).
  \end{itemize}
\end{exercise}

\begin{proof}
  For the first part notice that for any $U\in \Opens_X$ the set
  $\Hom(\{p\}\at{U},\sheaf{F}\at{U})(V)$ is given by the elements of
  $\sheaf{F}\at{U}(V)$. The isomorphism is thus given by
  $\Hom(\{p\}\at{X},\sheaf{F}\at{X})(V)\cong \sheaf{V}$.

  The similar argument holds in the second case, since the morphisms of
  $\underline{Z}\at{U}(V)\rightarrow \sheaf{F}\at{U}(V)$ corresponds
  to the subgroups generated by a unique element.

  The same argument holds for the last statement.
\end{proof}

\section{Compatible germs}

\begin{exercise}[2.4.A]
  See \cite[2.4.A]{vakil2024the-rising-sea}.
  The natural map
  \begin{align*}
    \sheaf{F}(U)\rightarrow \prod_{p\in U} \sheaf{F}_p
  \end{align*}
  is injective.
\end{exercise}

\begin{proof}
  Consider for each $U\in \Opens_X$ and $\sheaf{F}\in \category{C}_X$
  the set $\prod_{p\in U} \sheaf{F}_p$. This mapping
  \begin{align*}
    \sheaf{F}(U)\rightarrow \prod_{p\in U} \sheaf{F}_p
  \end{align*}
  is injective. For $f,g\in \sheaf{F}$ such that $f_p = g_p$ for all
  $p\in U$. Then there are open subsets $U_p\subset U$ and
  representatives $(\tilde{f},U_p)$ and $(\tilde{g},U_p)$ (do this by
  picking representatives and restricting to the intersection of their
  domain of definition) such that $\tilde{f}=\tilde{g}$. The set
  $\{U_p\}_{p\in U}$ is an open cover of $U$ and thus by the identity
  axiom of a sheaf $f=g$ holds.
\end{proof}

\begin{definition}
  We say that an element $(s_p)_{p\in U}$ of the right side
  $\prod_{p\in U} \sheaf{F}_p$ of the previous exercise consists of
  compatible germs if for all $p\in U$, there is some representative
  \begin{align*}
    (\tilde{s}_p \in \sheaf{F}(U_p), \; U_p \text{ open in }U);
  \end{align*}
  for $s_p$ (where $p\in U_p$) such that the germ of $\tilde{s}_p$ at
  all $q \in U_p$ is $s_q$. Equivalently, there is an open cover
  $\{U_i\}$ of $U$, and sections $f_i \in \sheaf{F} (U_i)$, such that
  if $p \in U_i$, then $s_p$ is the germ of $f_i$ at $p$. Clearly any
  section $s$ of $F$ over $U$ gives a choice of compatible germs for
  $U$.

  We denote the set of $(s_p)_{p\in U}\in \prod_{p\in U} \sheaf{F}_p$
  such that it consists of compatible germs by $\CGerms_{\sheaf{F}}(U)$.
\end{definition}

\begin{exercise}[2.4.B]
  See \cite[2.4.B]{vakil2024the-rising-sea}.
  Prove that any choice of compatible germs for a sheaf of sets
  $\sheaf{F}$ over $U$ is the image of a section of $F$ over $U$.
\end{exercise}

\begin{proof}
  For an element $(s_p)_{p\in U}\in
  \prod_{p\in U}\sheaf{F}_p$ in the set of compatible germs, we can
  for all $p\in U$ 
  find a representative  $\tilde{s}_p\in \sheaf{F}(U_p)$ for some
  $U_p$.
  The sets $U_p$ are an open cover of $U$.
  We need to check that for $U_q \cap U_p\neq \emptyset$ we have
  $\tilde{s}_p\at{U_q \cap U_p} = \tilde{s}_q\at{U_q \cap U_p}$. We
  see this, since for all $x \in U_q\cap U_p$ we have
  $(\tilde{s}_p\at{U_q \cap U_p})_x = (\tilde{s}_q\at{U_q \cap U_p})_x
  = s_x$. This means that there is a neigbourhood $V_x$ such that
  $\tilde{s}_p\at{V_x} = \tilde{s}_q\at{V_x}$ and these $V_x$ cover
  $U_p \cap U_q$. Thus applying the gluability, we get
  $\tilde{s}_p\at{U_q \cap U_p} = \tilde{s}_q\at{U_q \cap U_p}$. Thus
  we can again apply gluability to obtain a section
  $\tilde{s}\in \sheaf{F}(U)$.
\end{proof}

\begin{proposition}
  We can consider the functor defined by
  \begin{align*}
    \CGerms_\sheaf{F}\colon \Opens_X^\op \rightarrow \Sets, \;
    U\mapsto \CGerms_{\sheaf{F}}(U),
  \end{align*}
  An inclusion $\iota V\rightarrow U$ is mapped to the map
  $\prod_{p\in V}\pr_{p} $ of sets. This functor is
  naturally isomorphic to the sheaf $\sheaf{F}\in \Sets_X$.
\end{proposition}

\begin{proof}
  First notice that $\CGerms_\sheaf{F}$ is a functor at all. This
  holds since $\prod_{p\in U} \pr_{p} \prod_{p\in U}\{s_p\} =
  \prod_{p\in U}\{s_p\}$ and $\prod_{p\in W} \pr_{p} \prod_{p\in V}
  \pr_{p} = \prod_{p\in W} \pr_{p}$. We define the natural isomorphism
  \begin{align*}
    \eta_U \colon \CGerms_\sheaf{F} (U) \rightarrow \sheaf{F}(U)
  \end{align*}
  by
  mapping an element of compatible germs to the unique section defined
  in the previous exercise. The inverse for this is given by the
  canonical map $\sheaf{F}(U) \rightarrow \prod_{p\in U} \sheaf{F}_p$
  as we have seen in the previous exercises.

  Now to check that $\eta $ is a natural transformation:
  \begin{equation}
    \begin{tikzcd}[nodes in empty cells, column sep= 1.5cm, row sep=1.5cm]
      \CGerms_\sheaf{F} (U)
      \ar[r, "\prod_{p\in W} \pr_{p}" ]
      \ar[d, "\eta_U"]
      &
      \CGerms_\sheaf{F} (W)
      \ar[d, "\eta_W"]
      \\
      \sheaf{F}(U)
      \ar[r,"\res_{U,W}"]
      &
      \sheaf{F}(W)
    \end{tikzcd}
  \end{equation}
  For an element $(s_p)_{p\in U}$ there is a unique section $\eta_U
  ((s_p)_{p\in U}) \in \sheaf{F}(U)$ which we restrict to $W$. This
  restriction does not change the germs $\eta_U
  ((s_p)_{p\in U})_p$ for all $p\in W$.

  Thus by a similar argument to (2.4.B) we obtain $\eta_w(s_p)_{p\in
    W} =  \res_{U,W}\eta_U
  ((s_p)_{p\in U})$ which shows that $\eta$ is a natural isomorphism.
\end{proof}


%%% Local Variables:
%%% mode: latex
%%% TeX-master: "main"
%%% End:
