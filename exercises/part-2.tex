\chapter{Exercises Part 2}

\section{Projective varieties}

\begin{exercise}
  See \cite[Exercise 1.22]{görtz2010algebraic}.
  Let $L_1$ and $L_2$ be two disjoint lines in
  $\mathbb{P}^3(\field{k})$.
  \begin{enumerate}[label=(\roman*)]
  \item Show that there exists a change of coordinates such that $L_1 =
    V_+(X_0,X_1)$ and $L_2 = V_+(X_2,X_3)$.
  \item Let $Z = L_1 \cup L_2$. Determine the homogeneous radical ideal $\mathfrak{a}$ such that $V_+(\mathfrak{a})=Z$ (Exercise 1.21).
  \end{enumerate}
\end{exercise}

\section{Veronese embedding}

\begin{exercise}
  \cite[Exercise 1.30]{görtz2010algebraic}.
  Let $n, d > 0$ be integers. Let $M_0,\cdots ,M_N \in
  \field{k}[X_0,\cdots ,X_n]$ be all monomials in $X_0,\cdots ,X_n$ of
  degree $d$.
  \begin{enumerate}[label=(\roman*)]
  \item Define a $\field{k}$-algebra homomorphism
    \begin{align*}
      \theta \colon
      \field{k}[Y_0,\cdots,Y_N ] \rightarrow
      \field{k}[X_0,\cdots,X_n], \quad Y_i\mapsto  M_i
    \end{align*}
   and let $\mathfrak{a} =\ker \theta$. Show that $\mathfrak{a}$ is a
   homogeneous prime ideal (Exercise 1.20). Let
   $V_+(\mathfrak{a})\subset \mathbb{P}^N (\field{k})$ the projective
   variety defined by $\mathfrak{a}$ (Exercise 1.21).
 \item Consider the morphism
   \begin{align*}
     v_d \colon \mathbb{P}^n(\field{k}) \rightarrow \mathbb{P}^N
     (\field{k}),\quad(x_0 : \cdots : x_n) \mapsto
     (M_0(x_0,\cdots,x_n):\cdots : M_N (x_0,\cdots,x_n)),
   \end{align*}
   and show that $v_d$ induces an isomorphism $\mathbb{P}^n(\field{k})
   \cong V_+(\mathfrak{a})$ of prevarieties. Is $V_+(\mathfrak{a})$ a
   linear subspace of $\mathbb{P}^N (\field{k})$?
   \item Let $f \in \field{k}[X_0,\cdots ,X_n]$ be homogeneous of degree $d$. Show that $v_d(V_+(f ))$ is the intersection of $V_+(\mathfrak{a})$ and a linear subspace of $\mathbb{P}^N (\field{k})$. The morphism $v_d$ is called the $d$-Uple embedding or $d$-fold Veronese embedding.
  \end{enumerate}
\end{exercise}

\section{Spectra of rings}

\begin{exercise}
  Find two classes of rings or your favourite two specific rings and
  classify their spectrum.
\end{exercise}

\begin{exercise}
  See \cite[Exercise 2.9]{görtz2010algebraic}. For valuation rings see
  \cite[(B.13) (p562)]{görtz2010algebraic}.
  Let $\Gamma$ be a totally ordered abelian group. A subgroup $\Delta$
  of $\Gamma$ is called isolated if $0 \leq \gamma \leq \delta$ and
  $\delta \in \Delta$ implies $\gamma\in \Delta$.
  \begin{enumerate}[label=(\roman*)]
  \item Let $A$ be a valuation ring, $K =\Frac A$. Show that $\mathfrak{p}
    \mapsto A_{\mathfrak{p}}$ is an inclusion reversing bijection from
    $\Spec A$ onto the set of rings $B$ with $A \subset B \subset K$
    and that such a ring $B$ is a valuation ring of $K$. Its inverse
    is given by sending $B$ to its maximal ideal (which is contained
    in $A$).
  \item Let $A$ be a valuation ring with value group $\Gamma$. For
    every isolated subgroup $\Delta$ of $\Gamma$ set
    $\mathfrak{p}_\Delta \coloneqq \{  a\in A \mid v(a) \notin \Delta
    \}$. Show that $\Delta \mapsto \mathfrak{p}_\Delta$ defines a
    order reversing bijection between the set of isolated subgroups of
    $\Gamma$ and $\Spec A$ (both totally ordered by inclusion).
  \item Show that the value groups of the valuation ring
    $A_{\mathfrak{p}_\Delta}$ is isomorphic to $\Gamma/\Delta$ and the
    value groups of the valuation ring $A/\mathfrak{p}_\Delta$ is
    isomorphic to $\Delta$.
  \item For an arbitrary totally ordered abelian group $\Gamma$ let
    $R = \field{k}[\Gamma]$ be the group algebra of $\Gamma$ over
    some field $\field{k}$. Write elements of $u \in A$ as finite
    sums $u = \sum_{\gamma}a_\gamma e_\gamma$ . Define a map $v
    \colon R \setminus \{0\}\rightarrow \Gamma$ by sending $u$ to
    the minimal $\gamma \in \Gamma$ such that $\alpha_\gamma \neq
    0$. Show that $R$ is an integral domain and that $v$ can be
    extended to a valuation $v$ on $K =\Frac R$ with value group
    $\Gamma$. In particular $A \coloneqq \{ x\in K \mid v(x) \geq  0
    \}$ is a valuation ring with value group $\Gamma$.
  \item Let $I$ be a well-ordered set. Endow $\integers^I$ with
    the lexicographic order (i.e., we set $(n_i)_{i\in I} <
    (m_i)_{i\in I}$ if and only if $J\coloneqq \{ i\in I \mid m_i
    \neq n_i \}$ is non-empty and $n_{i_J} <m_{i_J}$ where $i_J$ is
    the smallest element of $J$). Show that $\integers^I$ is a
    totally ordered abelian group and that for all $k \in I$ the
    subsets $\Gamma_{\geq k}$ of $(n_i)_{i} \in \integers^I$ such
    that $n_i =0$ for all $i<k$ are isolated subgroups of
    $\integers^I$.
  \item Deduce that for every cardinal number $k$ there are
    valuation rings whose spectrum has cardinality $\geq k$.
  \end{enumerate}
\end{exercise}

\begin{proof}
  For $(i)$ consider the map $\mathfrak{p} \mapsto
  A_\mathfrak{p}$. This reverses orientation, i.e.\ if
  $q\subset p$ for two prime ideals, then $(A\setminus
  \mathfrak{p} )^{-1} = A_\mathfrak{p} \subset
  A_\mathfrak{q} = (A\setminus
  \mathfrak{q})^{-1}A$. This is also a valuation ring since for any
  $a\in \Frac A = K$ either $a\in A$ or $a^{-1}\in A$
  is naturally satisfied for $A\subset A_\mathfrak{p}$.

  To see that this is a bijection, we consider a local ring
  $A\subset B \subset K$, which we map to the preimage under the
  inclusion $A \subset B$ of its unique maximal ideal
  $\mathfrak{m}_B$. This is inverse to the localization above since
  \begin{align*}
    \mathfrak{m}_{A_\mathfrak{p}} = \mathfrak{p} A_\mathfrak{p}
  \end{align*}
  for a ring $A_\mathfrak{p}$ this is given by $\mathfrak{p}$.

  On the other hand for a local ring $A \subset B \subset K$ the
  localization at the preimage of the maximal ideal $\mathfrak{m}_B$
  has an inclusion into $B$ by the universal property of the
  localization. Since both of these rings are maximal with respect to
  domination and local they have to coincide.

  \vspace{\baselineskip}

  For $(ii)$ we can easily see that the set of isolated subgroups is
  totally ordered.
  Consider the map $\Delta \mapsto \mathfrak{p}_\Delta$. The image is
  a prime ideal since for $ab\in \mathfrak{p}_\Delta$ we have $v(ab) =
  v(a) + v(b) \notin \Delta$. This means that either $a$ or $b$
  already have to be in $\mathfrak{p}_\Delta$.

  We define the following inverse:
  \begin{align*}
    \mathfrak{p} \mapsto v(A \setminus \mathfrak{p})\cup -v(A \setminus \mathfrak{p}).
  \end{align*}
  This is a group since for $p,q\notin \mathfrak{p}$ we have $v(p)+
  v(q) = v(pq)\in v(A \setminus \mathfrak{p})$. We also need to check that it is isolated.
  Assume it is not isolated. Since $\mathfrak{p}$ is prime, the set
  $A\setminus \mathfrak{p}$ is multiplicatively closed. This means
  that $v(A \setminus \mathfrak{p})\cup -v(A \setminus \mathfrak{p})$
  is a group. On the other hand it contains any element between
  $-\alpha $ and $\alpha$ for $\alpha \in v(A \setminus \mathfrak{p})$
  since otherwise for $0\leq \beta \leq \alpha$ with $\beta \in
  v(\mathfrak{p})$ we would also have $v(A) + \beta \in
  v(\mathfrak{p})$. But since for $p \in \mathfrak{p}$ and $q\in A
  \setminus \mathfrak{p}$ with $v(p) = \beta $ and $v(q) = \alpha$ the
  element $v(p^{-1}q) = \alpha -\beta$ is in $v(A)$ or $-v(A)$ we
  would get that $\alpha \in v(\mathfrak{p})$ which is a contradiction. 
  
  We check the inverses:
  \begin{align*}
    v(\mathfrak{p}_\Delta) &= v(A \setminus \mathfrak{p}_\Delta)\cup - v(A \setminus \mathfrak{p}_\Delta) =
    \Delta\\
    \mathfrak{p}_{v(A\setminus \mathfrak{p})\cup - v(A \setminus \mathfrak{p})} &= \{  a\in A \mid v(a)
    \notin v(A\setminus \mathfrak{p}) \cup - v(A \setminus
                                                                                  \mathfrak{p}_\Delta )\} = \mathfrak{p} 
  \end{align*}

  \vspace{\baselineskip}

  For $(iii)$ the value group of $A$ the kernel of the group morphism
  $K^\times \rightarrow v(K^\times)$ is given by $A^\times$. Thus for
  the value group of $A_{\mathfrak{p}_\Delta}$ this kernel is given by
  $(A\setminus \mathfrak{p}_\Delta)$. Thus the map
  $v\colon K^\times / (A\setminus \mathfrak{p}_\Delta) \rightarrow
  v(K^\times)$ is an isomorphism with $K^\times / (A\setminus
  \mathfrak{p}_\Delta) \cong \Gamma /\Delta$ since $v(A\setminus
  \mathfrak{p}_\Delta) = \Delta$.

  \vspace{\baselineskip}

  For $(iv)$ consider the group algebra $R=\field{k}[\Gamma]$. Assume there are
  zero-divisors, i.e.\ $u = \sum_{\gamma} a_\gamma e_\gamma$ and
  $w=\sum_{\eta} b_\eta e_\eta$ with $u w = 0$. Then $v(u)=\gamma_0$
  and $v(w) = \eta_0$. Then $uw = \sum_{\eta ,\gamma} a_\gamma b_\eta
  e_{\gamma + \eta}$ where the summand
  $a_{\gamma_0}b_{\eta_0}e_{\gamma_0 + \eta_0}$ is
  non-zero since $\field{k}$ is integral.

  Extending $v$ to $K=\Frac R$ in the obvious way, i.e.\
  $v(\frac{u}{w}) = v(u)- v(w)$ we have that $v(K^\times) \cong
  \Gamma$ by definition. And $v$ fulfills
  \begin{itemize}
  \item $v(a b) = v(a) + v(b)$,
  \item $v(a + b) \geq \min \{v(a),v(b)\}$ with $>$ if and only if
    $v(a)=v(b)$ and the
    coefficients cancel,
  \item $v(0) = \infty $ by definition,
  \end{itemize}
  and is thus a valuation on $K$.
  
  \vspace{\baselineskip}

  For $(v)$ consider $(n_i)_{i\in I}$ and $(m_i)_{i\in I}$. If $J = \emptyset$
  we have $(n_i)=(m_i)$ and $(n_i)< (m_i)$ as well as $(n_i)> (m_i)$.
  Thus assume that $J\neq \emptyset$. Then there exists $i_J$ as a
  minimal element in $J$. Then, since $\integers$ are totally ordered,
  we get that $(m_i)$ and $(n_i)$ are comparable. Reflexivity,
  antisymmetry and transitivity are clear by total order in
  $\integers$.

  The structure of an abelian group in $\integers^I$ is just given by
  the pointwise group structure.

  The subsets $\Gamma_{\geq k}$ of $(n_i)_{i\in I}$ such that $n_i=0$
  for all $i<k$ are isolated.
  Let $0 \leq (n_i) \leq (m_i)$ for $(m_i)\in \Gamma_{\geq k}$. Then
  we have $n_j = 0$ for all $j<k$ since it is ``sandwiched''. Thus
  $(n_i)\in \Gamma_{\geq k}$.

  \vspace{\baselineskip}

  For $(vi)$ we just combine the previous parts. Thus from
  $(n_i)\in \Gamma_{\geq k}$$\field{k}[\integers^I]$ we get a
  $(n_i)\in \Gamma_{\geq k}$ valuation ring with spectrum cardinality
  $(n_i)\in \Gamma_{\geq k}$given by $\abs{I}$. 
  
\end{proof}

%%% Local Variables:
%%% mode: latex
%%% TeX-master: "main"
%%% End:
